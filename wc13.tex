\documentclass[a4paper]{exam}

\usepackage{amsmath,amssymb, amsthm}
\usepackage[a4paper]{geometry}
\usepackage{hyperref}

\theoremstyle{theorem}
\newtheorem{theorem}{Theorem}

\theoremstyle{claim}
\newtheorem{claim}{Claim}

\qformat{{\large\bf \thequestion. \thequestiontitle}\hfill}
\boxedpoints

\usepackage{draftwatermark}
\SetWatermarkText{Sample Solution}
\SetWatermarkScale{3}
\printanswers

\title{Weekly Challenge 13: Decidability}
\author{CS 212 Nature of Computation\\Habib University}
\date{Fall 2023}

\begin{document}
\maketitle

\begin{questions}

\titledquestion{Bit flip}

  Given $\Sigma=\{0,1\}$, consider the function, $f$, defined over a character, $a$, and extended to a string, $u=u_1u_2u_3\ldots u_n$, as follows.
  \begin{align*}
    f(a) & =
           \begin{cases}
             0 & a = 1\\
             1 & a = 0\\
           \end{cases}\\
    f(u) & = f(u_1)f(u_2)f(u_3)\ldots f(u_n)
  \end{align*}

  Assuming the results in Chapter 4, prove or disprove that the following language is decidable.
\[
  S = \{\langle M\rangle \mid M\text{ is a DFA},  u\in L(M) \implies f(u)\in L(M)\}.
\]

\begin{solution}
  We prove by construction that $S$ is decidable and by using the result that $EQ_{DFA}$ is decidable.

  For every string, $u$, that a DFA in $S$ accepts, it also accepts $f(u)$, the bit-flip version of $u$. That is, if the DFA accepts a binary sequence with a $0$ in particular position, then it accepts another binary sequence with a $1$ in that position, and vice versa. Every transition in the DFA that is labeled with $0$ must also be labeled with $1$, and vice versa. That is, every transition is labeled with both $0$ and $1$.

  We can check whether a DFA is of the above form by checking if it is equivalent to another DFA which has the same states and transitions but flipped transition labels.

  \begin{proof}
    Let $E$ be a decider for $EQ_{DFA}$.\\
    On input $\langle M \rangle$,
    \begin{itemize}
    \item Construct $N$ from $M$ by flipping the transition labels of $M$.

    \item Run $E$ on $\langle N, M\rangle$.
    \item If $E$ accepts, \textit{accept}; if $E$ rejects, \textit{reject}.
    \end{itemize}
  \end{proof}
\end{solution}

\end{questions}
\end{document}

%%% Local Variables:
%%% mode: latex
%%% TeX-master: t
%%% End:
